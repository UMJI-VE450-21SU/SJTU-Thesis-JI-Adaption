\chapter*{Yichao Yuan's Individual Contribution Report}

Our capstone design is ``RISC-V SoC Microarchitecture Design and Optimization''. In this capstone project, we designed a CPU based on RISC-V instruction set architecture (ISA). We investigate and use some advanced optimization techniques and build a software environment for verification and testing. During this process, each team members contribute to the processor design and optimization.

Designing a processor based on RISC-V ISA needs joint efforts from both the hardware part, which is about the design itself, and the software part, which helps verification and testing. In our team, Li Shi, Yiqiu Sun and I are hardware engineers, while Yichao Yuan and Zhiyuan Liu are software engineers. We hold the group meetings twice a week and co-design the hardware circuit design and software debugging workflow to build our RISC-V processor. Each member contributes to the processor design equally, and here is each member’s contribution to the project.

\begin{enumerate}
  \item Li Shi is responsible for hardware backend design, including instruction dispatch, issue units, scoreboard, physical register file and part of execution units. At the same time, he also participates in integrating the microarchitecture and debugging. His workload is about 15 hours per week.
  \item Yiqiu Sun is responsible for designing and integrating the overall microarchitecture. She provides technical support for the team and designs the instruction fetch unit and branch predictor. Meanwhile, she helps other team members review the code, raises many constructive suggestions, and improves the overall design quality. Her workload is about 15 hours per week.
  \item Jian Shi is responsible for designing the hardware frontend, including register renaming table, free list, re-order buffer. He also designs the floating-point units in the execution units in the backend. Besides, during the preparation stage of this project, He is partially responsible for the survey of open-source RISC-V core and SoC and read some academic articles. His workload is about 15 hours per week.
  \item Zhiyuan Liu is responsible for compilation workflow. She builds and modifies compiler toolchains, which is important for embedded system construction. She also adds our custom RISC-V instructions for approximate computing to the compiler so that users can use these custom instructions in their C programs. Her workload is about 15 hours per week.
  \item I am responsible for software simulation and validation. I help the team to validate the processor design in software design tools, including Xilinx Vivado, Verilator, Spike, etc. Besides, although not implemented in the final design, I explore the instruction and data cache design and AXI bus protocol for the SoC. His workload is about 15 hours per week.
\end{enumerate}

As previously mentioned, I am the software engineer in our team, and my job mainly focuses on building simulation and validation environment.

In terms of the technical part, it is my first time to get involved in the design of such a large digital system, and it is also the first time I participate in a design that needs effort from software level, architectural level and hardware level. During my undergraduate study, I have experience in all these levels, however, bringing them together and build a system from scratch is still challenging. I review a lot of concepts I learnt before and study in-depth about simulators and FPGA. During the first half of this project, I investigate building memory system with FPGAs. During the second half of this project, I focus on building an integrated simulation and validation system to verify the hardware design.

How to provide an environment for the computation core is an important point to considered in this system. The computation core needs at least some memory systems and IO system to be tested and verified. As this project aims at doing microarchitecture optimization, two domains can be considered: hardware implementation and software simulation. The former part provides very good performance and more solid result; However, it demands resources and provides little flexibility. In comparison, performing a software simulation is easy to implement, require nearly no resources and can be easily configured. I investigate the hardware implementation option first. However, the synthesize result shows it occupies too much resource and it is the bottleneck of the system. To better serve the overall goal of optimizing microarchitecture, I then turn to the software simulation path, and the system ties the software verification environment with the computation core successfully. I build the verification environment’s frontend server, which provides memory interfaces and IO interfaces, by building our own model as well as adapting part of the source code from the Spike simulator. I tune the compilation process so that the generated executable can be correctly loaded and executed by the models. I write libraries for this simulation system and implement some programs for test and demonstration using these libraries. I help organize part of the overall flow by writing makefiles. The final system boosts the verification cycle and demonstrates positive result of our CPU design.

In terms of the technique support part, I also learn a lot from this project. The project is organized by both Feishu and Git. The Feishu tracks members progress and help resolved the dependency among tasks. I report my progress on this system steadily to make sure I can cooperate well with my teammates. The Git holds repository of the codes of this project. During the first half of this project, I mainly working on the branch \texttt{Yuan/memory}, working on implementing the memory subsystem of this project on FPGA. After we turn to software simulation, we start to working on \texttt{Test/T1} and \texttt{Test/verilator} branch to work out the software stack of this project. This method makes code review possible and enhance cooperation.

In terms of the technical communication part, I participate in each design review, final report and final expo. In the design review 1, I am responsible for reviewing the problem of this project. In the design review 2, I am responsible for conducting the concept selection between static scheduling and dynamic scheduling. In the design review 3, I present the demo of the project. In the final expo, I take part in the oral presentation and JI design expo. I actively prepare the technical communication so that we can introduce our audience with this amazing project, and make sure that they can understand our design with ease as well as provide us valuable responses.

All in all, as a software engineer in this project, I contribute to the project of ``RISC-V SoC Microarchitecture Design and Optimization'' actively a lot and benefit from this experience a lot. As a team, we cooperate together and design a complex RISC-V processor with optimization.
