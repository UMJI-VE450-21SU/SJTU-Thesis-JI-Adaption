\chapter*{Yiqiu Sun's Individual Contribution Report}

Our capstone design is ``RISC-V SoC Microarchitecture Design and Optimization''. We design a processor based on RISC-V instructions set architecture (ISA). We simulate and verify the SoC in software simulator tools and further apply some advanced optimizations. This is a project that involves both hardware and software. All team members contribute to the processor design and optimization based on their expertise.

In our team, Jian Shi, Li Shi and I are hardware engineers, while Yichao Yuan and Zhiyuan Liu are software engineers. We hold the group meetings twice a week and co-design the hardware circuit design and software debugging workflow to build our RISC-V processor. Each member contributes to the processor design equally, and here is each member’s contribution to the project.

\begin{enumerate}
  \item During the preparation stage of this project, Jian Shi is partially responsible for the literature review for open-source RISC-V core and SoC. Jian Shi is responsible for designing the core hardware for Out-of-order pipeline, including register renaming table, free list, re-order buffer. He also works actively on the design of floating-point units (both accurate and approximate) in the execution stage. His workload is about 15 hours per week.
  \item Yichao Yuan is responsible for software simulation and validation. He is in charge of validating the processor design on various software design tools, such Xilinx Vivado, Verilator, Spike, etc. For every verification tool, he always designs a detailed plan. He also explores the instruction and data cache design and AXI bus protocol for the SoC although it is not implemented in the final design due to time. His workload is about 15 hours per week.
  \item Zhiyuan Liu is responsible for compilers. She builds and modifies compiler toolchains, which is important for embedded system construction. She also customizes the compiler by adding a new RISC-V instruction for approximate computing so that users can use these custom instructions in their C programs. Her workload is about 15 hours per week.
  \item Li Shi is responsible for hardware backend design. This includes instruction dispatch, issue units, scoreboard, physical register file and part of execution units. At the same time, he also actively participates in microarchitecture integration and debugging. His workload is about 15 hours per week.
  \item I am responsible for designing and integrating the overall microarchitecture. Meanwhile, I provide technical support for the team and design the instruction fetch module, fetch buffer module and branch predictor module. Besides that, I implement the int multiplier and divider for the execution stage. I also help other team members review the code by raising many constructive suggestions thus, improving the overall design quality. My workload is about 15 hours per week.
\end{enumerate}

Throughout this project, I am mainly in charge of the hardware. Before the start of the project, I already know the whole project is very challenging because I have taken EECS 470: Computer Architecture in University of Michigan, which is a similar but less interesting work than this capstone design. My previous course experience does help me better understand the goal and smoothen the concept selection process. But I have to be honest that the benefits are limited as our capstone design is more professional and advanced. For example, although I have designed a branch predictor before, when I am designing the new branch predictor for this capstone, I find new optimization points and there are still many new sophisticated details for me to consider. I keep reminding myself that ``Practice makes perfect'' and ``Devils are in the details''. Only by continuously optimizing the microarchitecture details can we achieve an overall best performance. Besides interacting with modules that I am familiar with, I also learn some new designs such as approximate computing. Those new designs are a perfect example of what engineering is. Engineering is not about finding a universally accepted truth, but about striking a balance between different considerations and applying optimizations in a specific scenario.

In terms of the technical communication part, I actively participate in the preparation for each design review and final expo. In design review 1, I am responsible for setting the timeline and overall workflow. In design review 2, I partially discuss our concept selection process. In design review 3, I present our final implement plan. In my bi-week meeting with advisor, I give a presentation about the branch predictor that we built. I also actively participate in our promotional video making process. I will participate in the oral defense and JI design expo in the final expo. In general, my goal is to not only design a high-performance processor, but also make more people understand what we have done and how they can utilize our processor to do great things in different fields. We also would like to inspire more and more young people to join us in the research of computer architecture.

This project also familiarizes me with various software that are necessary to remote-working. Due to COVID-19, I decide to stay in the US and participate the project remotely. I manage to pull through many difficulties that come with this choice, including time-zone difference, poor internet connection and schedule conflict.

I also learned a lot from my teammates. They are both professional and respectful. We meet frequently and have many meaningful debates. I always enjoy hearing different voices and discussing the pros and cons with them. We have a great atmosphere and strong team spirits.

In brief, as one of the hardware engineers in the team, I actively contribute to the project of ``RISC-V SoC Microarchitecture Design and Optimization''. We work as a team and design a complicated RISC-V processor. Although there is definitely some room to improve in the future, we have done our best to design and verify in such a short time.
