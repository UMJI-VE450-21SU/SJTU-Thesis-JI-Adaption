\chapter*{Jian Shi's Individual Contribution Report}

Our capstone project is about ``RISC-V SoC Microarchitecture Design and Optimization''. In this project, we first design an out-of-order RISC-V processor supporting 4-way superscalar execution and instruction dynamic scheduling. What’s more, we also try to optimize the SoC with approximate computing units. During the project, all team members make their own contribution to our design and we are satisfied about our final product.

To design and optimize a processor based on RISC-V ISA, we need a lot of efforts, from the hardware part, to the software part. In our group, Li Shi, Yiqiu Sun and I are responsible for hardware engineering, while Yichao Yuan and Zhiyuan Liu are responsible for software engineering. Every week, we hold the group meetings twice, on Wednesday and Saturday. The software part and the hardware part work together to build our own RISC-V processor. We roughly make the same contribution to the processor design. Each member’s contribution to the project is as follows.

\begin{enumerate}
  \item Li Shi focuses on hardware backend design, including instruction dispatch, issue units, scoreboard, physical register file and part of execution units. On the other hand, he also takes part in integrating the microarchitecture and debugging. His workload is roughly 15 hours per week.
  \item Yiqiu Sun focuses on designing and integrating the overall microarchitecture. She provides technical support for the team and designs the instruction fetch unit and branch predictor. What’s more, she helps other team members review the code, raises many constructive suggestions, and improves the overall design quality. Her workload is roughly 15 hours per week.
  \item Yichao Yuan focuses on software simulation and validation. He helps the team to validate the processor design in software design tools, including Xilinx Vivado, Verilator, Spike, etc. Besides, although not implemented in the final design, he explores the instruction and data cache design and AXI bus protocol for the SoC. His workload is roughly 15 hours per week.
  \item Zhiyuan Liu is focuses on compilation workflow. She builds and modifies compiler toolchains, which is important for embedded system construction. She also adds our custom RISC-V instructions for approximate computing to the compiler so that users can use these custom instructions in their C programs. Her workload is roughly 15 hours per week.
  \item I focus on designing the hardware frontend, including register renaming table, free list, re-order buffer. I also design the floating-point units in the execution units in the backend. Besides, during the preparation stage of this project, I am partially responsible for the survey of open-source RISC-V core and SoC and read some academic articles. My workload is roughly 15 hours per week.
\end{enumerate}

As previously mentioned, I am the hardware engineer in our team, and my job mainly focuses on register transfer level (RTL) design and optimization.

In terms of the technical part, it is my first time to design and test a System-on Chip (SoC), which means that I need to learn a lot of concepts in computer architecture, embedded systems and logic circuit design. On the other hand, in my four-year university study, I only have experience in a five-stage MIPS CPU design with in-order execution using Verilog language. Therefore, it is also my first time to take part in design for a processor core that supports execution and instruction dynamic scheduling with SystemVerilog language. For this reason, I studied a lot in digital design, and circuit simplification. In the evaluation stage, I made a survey for open-source RISC-V cores and SoC. Besides, to find design suitable for our SoC, I read many academic articles about approximate computing, floating point units and FPGA architecture.

One of the main parts in our project is the microarchitecture design, which requires us to select between concepts and make a balance in many aspects. For instance, the register renaming table needs to be recovered when mis-prediction or exceptions happen. However, there exist two structures to recover it accurately: checking point and retirement rename allocation table (rRAT). The former means that the processor should check the table when a branch instruction is renamed, while the latter just update the rRAT when an instruction is retired. In comparison, the method based on checking point requires a larger circuit but can provide faster recover speed. We first implement the first one in our processor. However, the final verification results show that a larger circuit leads to timing violation in FPGA implement. Therefore, I replace it with the rRAT one.

In terms of the technical support part, I also learn a lot from this project. We use my central server for circuit verification and implementation. The central server is much more powerful than our personal computers and can handle many computing tasks, such as compiling and logic synthesis. I am responsible for maintaining the server and provide in-time technique support. For example, I deploy the virtual network console (VNC) on the server so that we will share the same graphical user interface (GUI) during the project. Besides, I also teach my group mates to use Xilinx Vivado as a tool for logic synthesis and implementation. To share and manage our source code, we use Git and create an organization on GitHub. In \texttt{UMJI-VE450-21SU/Ria} repository on GitHub, I have my own working branches, including \texttt{Jian/Rename}, \texttt{Jian/ROB}, \texttt{Jian/FloatingPoint}, etc.

In terms of the technical communication part, we roughly have the same workload for each design review, final report and final expo. In design review 1, I do the literature search and take them as a benchmark for our design. In design review 2, I introduce the concept of instruction set architecture (ISA) and make a comparison between ARMv7 ISA and RISC-V 32G ISA. In design review 3, I compare our measured value with the target value in engineering specification. I also point out our design oversights in the engineering specification part. In the final expo, I take part in the oral presentation and JI design expo. Besides, I am responsible for the promotion video cutting. I devote myself to present an easy-to-understand and amazing project to the audience so that they can be inspired by our design and give us useful response for further improvement.

All in all, as a hardware engineer in this project, I have exploited my potential and make my own contribution to the project of ``RISC-V SoC Microarchitecture Design and Optimization''. We cooperate as a team and design a complex processor with RISC-V ISA.
