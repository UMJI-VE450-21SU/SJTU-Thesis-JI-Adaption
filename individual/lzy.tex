\chapter*{Zhiyuan Liu's Individual Contribution Report}

Our capstone design topic is ``RISC-V SoC Microarchitecture Design and Optimization'' and in this project, we propose a processor design that can support 4-way superscalar execution and instruction dynamic scheduling together with some advanced optimizations like approximate floating-point computing units to solve challenges in AI-oriented embedded system which requires the CPU to be energy-efficient, inexpensive and fast at same time. After designing our processor, we also set up the SoC and peripheral components like cache and I/O devices in software simulator tools to test and validate our design. Our project can be divided into hardware part and software part. Li Shi, Jian Shi and Yiqiu Sun are mainly hardware engineers in our team while Yichao Yuan and I are software engineers. We hold our group meetings two to three times a week and there are also some individual meetings. Besides, we have meetings with our instructor every two weeks to report our project progress and explain some technical issues. Everyone has done their best for our design and everyone has contributed to our project equally, the following is our division of work.

\begin{enumerate}
  \item Li Shi: He is responsible for most of backend designs like instruction dispatch, instruction issue, register file and execution units for integer and memory access. He also helps debug and integrate FP components. His workload is about 15 hours per week.
  \item Jian Shi: He is responsible for hardware frontend designs such as free list, register renaming table, reorder buffer and execution units for floating point. At the same time, during the preparation phase of our project, he also did a lot of research on other RISC-V cores. His workload is about 15 hours per week.
  \item Yiqiu Sun: She is responsible for designing the branch predictor and instruction fetch unit and helps integrate and design our overall microarchitecture. She is also our technical support and helps review our code and proposes many constructive ideas about our design. Her workload is about 15 hours per week.
  \item Yichao Yuan: He is responsible for software simulation and validation part. He helps to replicate the Spike-based model to our verilator-based model, so that our CPU core can be better compared with the Spike model, and help simulation and validation. He also explores cache design and AXI bus protocol for our project at the preparation stage. His workload is about 15 hours per week.
  \item Zhiyuan Liu: I am responsible for the compilation workflow and I also do parts of the instruction fetch at the beginning of our project with the help of Yiqiu Sun. I investigate and study the structure of LLVM compiler and GCC compiler and try to modify and rebuild the compiler toolchains such that customized instructions for approximate computing units can be issued and disassembled by our new compiler toolchains. I also integrate the approximate computing functions into Spike to help validate our core. My workload is about 15 hours per week.
\end{enumerate}

As mentioned in the previous part, I am the software engineer in our team, and my job is mainly concentrated on compilation workflow and Spike validation part. But at the beginning of the project, I am also responsible for part of the hardware frontend design - instruction fetch. I have had the experience of RTL design, but I have not been in touch with SystemVerilog before, so it is painful when I first start writing SystemVerilog for our project. In the beginning, I just treat it purely as an ordinary software language, but in the process of writing and with the help of our team members, I can now roughly understand how to use SystemVerilog to describe the behavior of the hardware. With the help of Yiqiu Sun, we complete the instruction fetch part. In the middle and late part of our project, I focus on the compilation workflow. In this process, I compare the architecture of the LLVM compiler with that of GCC, and get an in-depth understanding of the compilation process. Between LLVM and GCC compiler, after comparing their structure, I select GCC compiler. Among many versions of RISC-V GCC compiler, I finally choose riscv64-multilib-elf-gcc as our target compiler because the design goal of our CPU is for embedded system, so we must use static link library, and we want our design to be better compatible with 64-bit RISC-V architecture in the future. At the same time, I also study how to modify and rebuild the compiler so that it can issue and disassemble our customized instructions for approximate computing, how to add our customized instruction to Spike, and how to describe the functional behavior of our customized instruction so that Spike can also run approximate computation for floating point.

I also learn a lot about how to manage engineering project in our capstone design. 

Many useful project management tools are used in our project such as Feishu and Git. We use Feishu to arrange our group meetings and use Feishu docs to track each team member’s progress and schedule. We use Git to manage our code. For example, we have a Feishu docs named ``VE450 project management''. In that docs, we record the division of work for everyone and update our progress in time.

In terms of the technical communication part, I actively participate in each presentation and our final expo. In our presentations, I am responsible for introducing the background of our project to the audience, the actual problems that our project solved, and the innovations of our project. I also make a technical report and presentation on compilation workflow to our instructor.

In conclusion, as one of the software engineers in our team, I try my best to contribute to our capstone design. We work as a team and everyone in our group tries their best to make the project better.
