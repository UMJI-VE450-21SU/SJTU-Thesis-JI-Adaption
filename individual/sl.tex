\chapter*{Li Shi's Individual Contribution Report}

In our capstone design of ``RISC-V SoC Microarchitecture Design and Optimization'', based on RISC-V instructions set architecture (ISA), we design a processor, apply some advanced optimizations, and set up the SoC in software simulator tools. During this process, each team members contribute to the processor design and optimization.

To design a processor based on RISC-V ISA requires the efforts from not only the hardware part, but also from the software part. In our team, Jian Shi, Yiqiu Sun and I are hardware engineers, while Yichao Yuan and Zhiyuan Liu are software engineers. We hold the group meetings twice a week and co-design the hardware circuit design and software debugging workflow to build our RISC-V processor. Each member contributes to the processor design equally, and here is each member’s contribution to the project.

\begin{enumerate}
  \item Jian Shi is responsible for designing the hardware frontend, including register renaming table, free list, re-order buffer. He also designs the floating-point units in the execution units in the backend. Besides, during the preparation stage of this project, he is partially responsible for the survey of open-source RISC-V core and SoC and reads some academic articles. His workload is about 15 hours per week.
  \item Yiqiu Sun is responsible for designing and integrating the overall microarchitecture. She provides technical support for the team and designs the instruction fetch unit and branch predictor. Meanwhile, she helps other team members review the code, raises many constructive suggestions, and improves the overall design quality. Her workload is about 15 hours per week.
  \item Yichao Yuan is responsible for software simulation and validation. He helps the team to validate the processor design in software design tools, including Xilinx Vivado, Verilator, Spike, etc. Besides, although not implemented in the final design, he explores the instruction and data cache design and AXI bus protocol for the SoC. His workload is about 15 hours per week.
  \item Zhiyuan Liu is responsible for compilation workflow. She builds and modifies compiler toolchains, which is important for embedded system construction. She also adds our custom RISC-V instructions for approximate computing to the compiler so that users can use these custom instructions in their C programs. Her workload is about 15 hours per week.
  \item I am responsible for hardware backend design, including instruction dispatch, issue units, scoreboard, physical register file and part of execution units. At the same time, I also participate in integrating the microarchitecture and debugging. My workload is about 15 hours per week.
\end{enumerate}

As mentioned in the previous part, I am the hardware engineer in our team, and my job is mainly concentrated on register transfer level (RTL) design and optimization. 

In terms of the technical part, it is my second time to design a CPU core with hardware design language, and it is also the first time that I participate in designing a large processor core that supports superscalar execution and instruction dynamic scheduling. During this process, I have learned many basic concepts in computer architecture, digital design, and embedded systems. During the preparation stage, I investigate many open-source RISC-V cores and compare them in terms of performance, cost, and energy efficiency. After the project starts, I deep dive into the hardware design part. 

The microarchitecture design part is one of the core parts in our project and requires many concept selections and trade-off in the detailed designs. For example, in terms of the design of instruction dispatch stage, we have two options: collapsing-style dispatch and non-collapsing-style dispatch. The former means that we need to compress the incoming instructions and send to the issue stage, while the latter just selects the instructions and marks the unnecessary instruction as invalid. After carefully comparing the two design styles, I choose to implement the first one, which is more complicated, but can reduce the logic complexity in the issue stage. Later in the issue and register file stage, we also have two options: whether we should first read from the register file and then pass the values to the issue stage, or first issue the instructions and then read from the register file. The former may lead to higher efficiency, but we need to store all the source operand values in the issue units and frequently pass these values. On the contrary, the latter can save such unnecessary operations, save some power, and fits the customer requirements in embedded systems. Thus, after this process of analysis and referring to other open-source RISC-V cores, I choose to implement the latter design. We need to carefully compare the designs and choose the best fit for our processor.

In terms of the engineering project management part, I also learn a lot during the process of this capstone project. We apply many useful project management tools, including Feishu docs, which is useful to draw Gantt chart and track each group member’s schedule, and Git, which helps the team to manage the project repository as well as review the source code. For example, in the Feishu docs page of ``VE450 project management'', I record my progress and plan on a weekly basis, which will be automatically shown in the generated Gantt chart. In the group meeting, I report my weekly progress and share my hardware design or workflow with my team members. In the Git repository, I create my own working branches, including \texttt{Li/IssueQueue}, \texttt{Li/Dispatch}, \texttt{Li/ExecutionUnit}, etc., so that other people can easily check my progress and personal contribution to the project.

In terms of the technical communication part, I actively participate in the preparation for each design review and final expo. In design review 1, I am responsible for present our customer requirements and engineering specifications. In design review 2, I demonstrate our first version of processor design. In design review 3, I present our final microarchitecture design which consists of 9 pipeline stages in the frontend and backend. In the final expo, I participate in the oral defense and JI design expo. I try my best to not only work hard on the project itself, but also present our project to the audience so that people can clearly understand what we have done and how they can utilize our processor to do great things.

In brief, as one of the hardware engineers in the team, I actively contribute to the project of ``RISC-V SoC Microarchitecture Design and Optimization''. We work as a team and design a complicated RISC-V processor.
